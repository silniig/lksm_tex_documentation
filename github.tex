\documentclass[oneside, final, 12pt]{article}
\usepackage[utf8]{inputenc}
\usepackage[russianb]{babel}
\usepackage{vmargin}
\setpapersize{A4}
\setmarginsrb{2cm}{1.5cm}{1cm}{1.5cm}{0pt}{0mm}{0pt}{13mm}
\usepackage{indentfirst}
\usepackage{amsmath}
\usepackage{graphicx}
\sloppy
\begin{document}
\begin{center}
\Large{Руководство по работе с github для комсомольца}\\
\end{center}

\tableofcontents

\section*{Введение}
Данный текст не является исчерпывающим и содержит минимальный набор знаний, необходимый для начала работы с github. Возможно внесение дополнений.
Текущая версия создана \today
\section{Подготовка к работе}
\subsection{Создание учётной записи}
Для работы на github {\bf необходимо} создать учётную запись. Приглашение создать учётную запись находится на главной странице \texttt{github.com}. Рекомендуется выбирать легко запоминающийся и быстро набираемый пароль: в ряде случаев его придётся часто вводить. Создавать, конечно же, нужно бесплатную учётную запись.
\subsection{Настройка компьютера и установка ПО}
Возможна работа как на ОС \texttt{Windows}, так и на \texttt{Unix}-подобных ОС. Данное руководство ориентировано прежде всего на \texttt{Unix}-подобные системы.

Для \texttt{Unix}-подобных систем на примере \texttt{Debian}:
\begin{enumerate}
\item Установить текстовый редактор \texttt{vim}: \texttt{apt install vim}
\item Установить \texttt{git}: \texttt{apt install git}
\item Опционально: создать ключи для работы с \texttt{github} через \texttt{ssh}. {\bf Руководство по созданию будет добавлено позднее}
\item Опционально: пройти краткий курс по работе в \texttt{vim} с помощью \texttt{vimtutor}.
\end{enumerate}
Вышенаписанное проверялось на актуальном дистрибутиве \texttt{Debian} и \texttt{Ubuntu 18.04 LTS}. Если \texttt{apt} требует привилегий суперпользователя, добавьте в начало команды \texttt{sudo}: \texttt{sudo apt install <название пакета>}.

Для \texttt{Windows} без \texttt{VirtualBox}:
\begin{enumerate}
\item Установить официальный клиент \texttt{git} c сайта \texttt{git-scm.com}
\item {\bf Здесь позже будет подробный алгоритм установки с выбором конкретных опций установки}
\end{enumerate}

Для \texttt{Windows} с применением \texttt{VirtualBox}:
\begin{enumerate}
\item Установить \texttt{VirtualBox}
\item Создать виртуальную машину под управлением ОС \texttt{Debian} ({\bf подробное руководство будет добавлено позже})
\item Далее действовать по плану для \texttt{Debian}
\end{enumerate}

\section{Собственно, работа}
Работа с репозиторием делится на несколько этапов:
\subsection{Создание репозитория}
На странице со списком репозиториев (\texttt{github.com} после входа в учётную запись) в левом меню кнопка \texttt{New}. В открывшемся меню нужно придумать название и выбрать, будет репозиторий личным или публичным.
\end{document}