\documentclass[oneside, final, 12pt]{article}
\usepackage[utf8]{inputenc}
\usepackage[russianb]{babel}
\usepackage{vmargin}
\setpapersize{A4}
\setmarginsrb{2cm}{1.5cm}{1cm}{1.5cm}{0pt}{0mm}{0pt}{13mm}
\usepackage{indentfirst}
\usepackage{amsmath}
\usepackage{graphicx}
\sloppy
\begin{document}
\begin{center}
\Large{Руководство по работе с github для комсомольца}\\
\end{center}

\tableofcontents

\section*{Введение}
Данный текст не является исчерпывающим и содержит минимальный набор знаний, необходимый для начала работы с github. Возможно внесение дополнений.
Текущая версия создана \today
\section{Создание учётной записи}
Для работы на github {\bf необходимо} создать учётную запись. Приглашение создать учётную запись находится на главной странице \texttt{github.com}. Рекомендуется выбирать легко запоминающийся и быстро набираемый пароль: в ряде случаев его придётся часто вводить. Создавать, конечно же, нужно бесплатную учётную запись.
\section{Собственно, работа}
Работа с репозиторием делится на несколько этапов:
\subsection{Создание репозитория}
На странице со списком репозиториев (\texttt{github.com} после входа в учётную запись) в левом меню кнопка \texttt{New}. В открывшемся меню нужно придумать название и выбрать, будет репозиторий личным или публичным.
\end{document}